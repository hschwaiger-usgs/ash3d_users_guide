\chapter{Finite Volume Solvers}\label{ChapAppendFVSolvers}
\section{Grid Geometry}
%\subsection{Topography}\label{FV_Grid_Topo}
%https://nilu.brage.unit.no/nilu-xmlui/bitstream/handle/11250/2761899/08-2002-lhs.pdf
In Ash3d, the equations are implemented using a generalized coordinate system, using
spatially variable cell volumes and cell interfaces.

\subsection{Cell metrics}
For Cartesian grids, the volume ($\kappa$) and cell-interface areas
($\sigma_x$, $\sigma_y$, $\sigma_z$) are simple products of $\Delta x$,
$\Delta y$, and $\Delta z$. For spherical coordinates, we calculate cell volume as:
\begin{equation}\label{EqCellVolLL}
\kappa_{i,j,k} = \int_{r_K}^{r_{K+1}} \int_{\theta_J}^{\theta_{J+1}} \int_{\lambda_I}^{\lambda_{I+1}}
\, r \sin{\theta} \mathrm{d}\lambda \, r \mathrm{d}\theta \, \mathrm{d}r
= \left( \lambda_{I+1} - \lambda_{I} \right) \frac{1}{3} \left( r_{K+1}^3 - r_{K}^3 \right)
\left( \cos{\theta_{J}} - \cos{\theta_{J+1}}\right)
\end{equation}
The interface surface area is expressed as:
\begin{equation}\label{EqCellSurfILL}
\sigma_x=\sigma_{I,j,k} = \int_{r_K}^{r_{K+1}} \int_{\theta_J}^{\theta_{J+1}}
\, r \mathrm{d}\theta \, \mathrm{d}r
= \left( \theta_{J+1} - \theta_{J} \right) \frac{1}{2} \left( r_{K+1}^2 - r_{K}^2 \right)
\end{equation}
%
\begin{equation}\label{EqCellSurfJLL}
\sigma_y=\sigma_{i,J,k} = \int_{r_K}^{r_{K+1}} \int_{\lambda_I}^{\lambda_{I+1}}
\, r \sin{\theta} \mathrm{d}\lambda \, \mathrm{d}r
= \sin{\theta}\left( \lambda_{I+1} - \lambda_{I} \right) \frac{1}{2} \left( r_{K+1}^2 - r_{K}^2 \right)
\end{equation}
%
\begin{equation}\label{EqCellSurfKLL}
\sigma_z=\sigma_{i,j,K} = \int_{\lambda_I}^{\lambda_{I+1}} \int_{\theta_J}^{\theta_{J+1}}
\, r \sin{\theta} \mathrm{d}\lambda \, r\mathrm{d}\theta
= r_{K}^2 \left( \lambda_{I+1} - \lambda_{I} \right) \left( \cos{\theta_{J}} - \cos{\theta_{J+1}}  \right)
\end{equation}
In both the Cartesian and spherical coordinates, the cell length along an advection
or diffusion direction is expressed as the ratio of the perpendicular
surface area to the cell volume (e.g. $\Delta x = \frac{\kappa}{\sigma_z}$).
When topography is included, these metrics must be scaled by the Jacobian of
the transformation.
\section{Advection Routines}
Ash3d has been written to accommodate different advection routines such as
Donor-Cell-Update (DCU), Corner Transport Update (CTU), and Semi-Lagrange (SL).
In this public version of the software, only DCU is included.

The implementation of DCU follows closely with the equations presented in
\cite{LeVeque2003}.

Start with the fluctuation form (Eq. 6.59 from \cite{LeVeque2003}) of the update in $Q$ to
a cell $i$ given below.
\begin{equation}\label{EqQupdate}
Q^{n+1}_i = Q^{n}_i - \frac{\Delta t}{\Delta x} 
\left( \mathcal{A}^{-} \Delta Q_{i+\frac{1}{2}} + \mathcal{A}^{+} \Delta Q_{i-\frac{1}{2}}\right) -
\frac{\Delta t}{\Delta x}
\left( \widetilde{F}_{i+\frac{1}{2}} - \widetilde{F}_{i-\frac{1}{2}}\right)
\end{equation}
Note that this is analogous to the flux form with
$(A^{-}\Delta Q_{i+\frac{1}{2}} + A^{+}\Delta Q_{i-\frac{1}{2}})$ with the left and
right fluctuations at an interface satisfying Eq. 6.57 
\begin{equation}\label{Eqflucdef}
(\mathcal{A}^{-}\Delta Q_{i-\frac{1}{2}} + \mathcal{A}^{+}\Delta Q_{i-\frac{1}{2}})=
f(Q_i)-f(Q_{i-1})= A \Delta Q_{i-\frac{1}{2}}
\end{equation}
We define $\mathcal{A}^{-}$ and $\mathcal{A}^{+}$ according to Eq. 9.38 for the
conservative variable coefficient case.
\begin{eqnarray}
\mathcal{A}^{+} \Delta Q_{i-\frac{1}{2}} &=& F_i - F_{i-\frac{1}{2}} \\
\mathcal{A}^{-} \Delta Q_{i-\frac{1}{2}} &=& F_{i-\frac{1}{2}} - F_{i-1}
\end{eqnarray}
where $F_{i-\frac{1}{2}}$ are the interface fluxes and $F_i$ are the cell-centered
representations.  These cell-centered values are somewhat arbitrary since they cancel
each other in Eq. \ref{EqQupdate}, but we use
\begin{equation}\label{Eqccfluxdef}
F_i = (u^{+}_{i-\frac{1}{2}} + u^{-}_{i+\frac{1}{2}})Q_i
\end{equation}
as suggested by
Eq 9.39 of \cite{LeVeque2003} so that the cell-centered flux values are an approximation of
the flux in cell $i$.  Interface fluxes are given by (the un-numbered
equation of \cite{LeVeque2003} between 9.35 and 9.36):
\begin{equation}\label{EqIntfluxdef}
F_{i-\frac{1}{2}} = u^{+}_{i-\frac{1}{2}} Q_{i-1} + u^{-}_{i-\frac{1}{2}} Q_{i}
\end{equation}
So
\begin{eqnarray}
\mathcal{A}^{+} Q_{i-\frac{1}{2}} &=&  
  \left( u^{+}_{i-\frac{1}{2}} Q_{i}   + u^{-}_{i+\frac{1}{2}} Q_{i} \right) -
  \left( u^{+}_{i-\frac{1}{2}} Q_{i-1} + u^{-}_{i-\frac{1}{2}} Q_{i} \right) \\
                                  &=& 
 u^{+}_{i-\frac{1}{2}} \left( Q_{i} - Q_{i-1} \right) +
                 Q_{i} \left( u^{-}_{i+\frac{1}{2}} - u^{-}_{i-\frac{1}{2}}  \right)
\end{eqnarray}
\begin{eqnarray}
\mathcal{A}^{-} Q_{i-\frac{1}{2}} &=&  
  \left( u^{+}_{i-\frac{1}{2}} Q_{i-1} + u^{-}_{i-\frac{1}{2}} Q_{i}   \right) -
  \left( u^{+}_{i-\frac{3}{2}} Q_{i-1} - u^{-}_{i-\frac{1}{2}} Q_{i-1} \right) \\
                                  &=& 
  u^{-}_{i-\frac{1}{2}} \left( Q_{i} - Q_{i-1} \right) +
                Q_{i-1} \left( u^{+}_{i-\frac{1}{2}} - u^{+}_{i-\frac{3}{2}}  \right)
\end{eqnarray}
Note that the choice of the cell-centered flux given in Eq. \ref{Eqccfluxdef},
leads to fluctuation forms above that have components that represent the $u q_x$
as well as the $q u_x$ terms.
If topography is included, the advection-diffusion equation additionally contains
spatial derivatives of the topography. As noted in Ap. \ref{ChapAppendTopo}, we neglect spatial
derivitives of topography in the diffusion term. For the $z$-shifted case, the gradients of
topography are included only in the redefinition of the vertical velocity (See
Eq.s \ref{v_sub1} and \ref{AdvDif_topo1}), with horizontal advection and diffusion
along $s = $ constant and vertical perpendicular to constant $s$ surfaces. For the
$\sigma$-altitude case, vertical velocity is redefined according to Eq. \ref{v_sub2},
but the horizontal advection equation does include gradients of topography
(Eq. \ref{AdvDif_topo2}). For this case, we modify the cell-centered flux (Eq. \ref{Eqccfluxdef})
to include the third term on the RHS of Eq. \ref{AdvTopo2X}:
\begin{equation}\label{EqccfluxdefTopo}
F_i = (u^{+}_{i-\frac{1}{2}} + u^{-}_{i+\frac{1}{2}})Q_i + \frac{u_i q_i}{2D_i}
\left( D_{i+1}-D_{i-1}\right)
\end{equation}
where topography is included via the variable $D_i=z_{top}-z_{surf}$, i.e. the depth of the model.

The higher order terms of Eq \ref{EqQupdate} above are given by Eq. 6.60 of \cite{LeVeque2003}.
\begin{equation}\label{EqHOTLeVeque}
\widetilde{F}_{i-\frac{1}{2}} = \frac{1}{2}\sum_{p=1}^{m} \left| s^p_{i-\frac{1}{2}}  \right|
\left( 1 - \frac{\Delta t}{\Delta x} \left| s^p_{i-\frac{1}{2}}  \right|  \right)
\widetilde{W}^p_{i-\frac{1}{2}}
\end{equation}
where $\widetilde{W}^p_{i-\frac{1}{2}}$ are the limited $p$-waves from an interface
and $s^p_{i-\frac{1}{2}}$ are their speeds.  In our case, we have $p=1$, 
$s^p_{i-\frac{1}{2}}=u_{i-\frac{1}{2}}$ and $\widetilde{W}_{i-\frac{1}{2}}$ is the limited version of
the wave, .i.e $\widetilde{\Delta Q}$.
Eq. \ref{EqHOTLeVeque} reduces to
\begin{equation}\label{EqHOTAsh3d}
\widetilde{F}_{i-\frac{1}{2}} = \frac{1}{2} \left| u_{i-\frac{1}{2}}  \right|
\left( 1 - \frac{\Delta t}{\Delta x} \left| u_{i-\frac{1}{2}}  \right|  \right)
\widetilde{\Delta Q}_{i-\frac{1}{2}}
\end{equation}
The limited wave, $\widetilde{\Delta Q}$, can be set by first determining the
function
\begin{equation}
\theta = \frac{\Delta Q_{\mathrm{upwind}}}{\Delta Q_{i-\frac{1}{2}}}
\end{equation}
then setting
$\widetilde{\Delta Q} = \phi\left( \theta_{i-\frac{1}{2}} \right) \Delta Q_{i-\frac{1}{2}}$
(Eq. 6.34 of \cite{LeVeque2003}):
\begin{itemize}
 \item No limiter: $\phi\left( \theta \right) = 0  \Rightarrow \widetilde{\Delta Q}_{i-\frac{1}{2}} = 0$
       (Note: this sets $\widetilde{F}_{i-\frac{1}{2}} = 0$)
 \item Lax-Wendrof: $\phi\left( \theta \right) = 1 \Rightarrow \widetilde{\Delta Q}_{i-\frac{1}{2}} = ( Q_i - Q_{i-1})$
 \item Beam-Warming:  $\phi\left( \theta \right) = \theta \Rightarrow \widetilde{\Delta Q}_{i-\frac{1}{2}} = \Delta Q_{\mathrm{upwind}}$
 \item Fromm: $\phi\left( \theta \right) = \frac{1}{2}\left( 1 + \theta\right) $
 \item MinMod: $\phi\left( \theta \right) = \mathrm{minmod}\left( 1,\theta \right)$
 \item SuperBee: $\phi\left( \theta \right) = \mathrm{max}\left(0, \mathrm{min \left( 1,2\theta\right),\mathrm{min}\left(2,\theta \right)} \right )$
 \item MC: $\phi\left( \theta \right) = \mathrm{max}\left(0,  \mathrm{min} \left( \frac{\left( 1+\theta \right)}{2} ,2,2\theta \right) \right) $
\end{itemize}

To allow curvilinear coordinates, we replace the $\Delta x$ in the above equations with $\kappa_i$
and scale velocities by the interface 
area $\sigma_{i-\frac{1}{2}}$, since these velocities are only needed at the interfaces.

\subsection{Numerical implementation}

Variables are generally associated with cell-centers and are indexed by $i$.  These include
$Q$, $u$, and $\kappa$.  In the equations above, several variables are needed on interfaces.
We use the following indexing for interface $I=i-\frac{1}{2}$ and define
\begin{align*}
\sigma_I &= \text{area of interface} &  \\
\bar{u}_I &= u_{i-\frac{1}{2}} &= \frac{1}{2}\left( u_i + u_{i-1} \right)\\
\Delta Q_I &= \Delta Q_{i-\frac{1}{2}} &= Q_i - Q_{i-1} 
\end{align*}
On the first loop over $I$, the following are calculated:
\begin{eqnarray*}
 \texttt{ubar\_I(1:ncells+1)} &=& \bar{u}_I \\
 \texttt{usig\_I(1:ncells+1)} &=& (\bar{u} \sigma)_I \\
 \texttt{dqi\_I(1:ncells+1)} &=& \Delta Q_I \\
 \texttt{ldq} &=& \widetilde{\Delta Q}_I \\
 \texttt{fss\_I(1:ncells+1)} &=& \widetilde{F}_I 
\end{eqnarray*}
where
\begin{equation}
\widetilde{F}_I = \frac{1}{2} \left| \bar{u}_I \right| 
\left( 1 - \left| (\bar{u} \sigma)_I \right| \frac{\Delta t}{\kappa_i}\right)
\widetilde{\Delta Q}_I
\end{equation}
On the second pass over $I$, $\mathcal{A}^{\pm} \Delta Q_I$ are calculated according to
\begin{eqnarray*}
\texttt{fs\_I(1:ncells+1,1)} = \mathcal{A}^{-} Q_I &=& \bar{u}_I^{-} \Delta Q_I  + 
Q_{i-1} (\bar{u}_I^{+} - \bar{u}_{I-1}^{+}) \\
\texttt{fs\_I(1:ncells+1,2)} = \mathcal{A}^{+} Q_I &=& \bar{u}_I^{+} \Delta Q_I  + 
Q_{i} (\bar{u}_{I+1}^{-} - \bar{u}_{I}^{-})
\end{eqnarray*}
Note that the first term of these equations uses the interface concentration
jump and left/right interface concentration while the second terms using cell-centered
concentrations multiplied by the differences in interface velocities across that
cell-centered value.  Therefore, calculating \texttt{fs\_I} at interface $I$ requires
cell-centered velocities at $i-2,i-1,i,i+1,i+2$.  Similarly, some limiters also require
concentrations on these cells.

Lastly, we loop over the cell index $i$ and build the four components of the update:
\begin{itemize}
 \item Rightward fluctuation at left boundary = \texttt{RFluct\_Lbound} = $\mathcal{A}^{+} \Delta Q_I$
 \item Leftward fluctuation at right boundary = \texttt{LFluct\_Rbound} = $\mathcal{A}^{-} \Delta Q_{I+1}$
 \item Limited-q at left boundary = \texttt{LimFlux\_Lbound} = $\widetilde{F}_I$
 \item Limited-q at right boundary = \texttt{LimFlux\_Rbound} = $\widetilde{F}_{I+1} $
\end{itemize}


\section{Diffusion Routines}
Ash3d uses two implementations of the diffusion term: the explicit finite difference
discretization and the implicit Crank-Nicolson method.  

\subsection{Explicit finite difference}
Using Eq. 4.11 of \cite{LeVeque2003}:
\begin{equation}\label{EqDiffFD_LV}
Q^{n+1}_i = Q^{n}_i + \frac{\Delta t}{\Delta x^2} \left[ k_{i+\frac{1}{2}}\left( Q^{n}_{i+1}-Q^{n}_{i}\right) -
k_{i-\frac{1}{2}}\left( Q^{n}_{i}-Q^{n}_{i-1}\right) \right]
\end{equation}
Again, we use the following indexing for interface $I=i-\frac{1}{2}$ and define
\begin{align*}
\sigma_I &= \text{area of interface} &  \\
\bar{k}_I &= k_{i-\frac{1}{2}} &= \frac{1}{2}\left( k_i + k_{i-1} \right)\\
\Delta Q_I &= \Delta Q_{i-\frac{1}{2}} &= Q_i - Q_{i-1} 
\end{align*}
Similar to the advection routines, we generalize Eq. \ref{EqDiffFD_LV} to curvilinear coordinates by replacing
$\Delta x$ with $\kappa$ and scaling the interface $\Delta Q_I$ by the square of the
interface area $\sigma_I^2$ leading to
\begin{equation}\label{EqDiffFD}
Q^{n+1}_i = Q^{n}_i + \frac{\Delta t}{\kappa_i^2} \left[ 
\left( k \sigma^2 \Delta Q^n \right)_{I+1} - 
\left( k \sigma^2 \Delta Q^n \right)_I  \right]
\end{equation}
Note that this method is stable when $\Delta t = \mathcal{O}(\Delta x^2)$

\subsection{Implicit Crank-Nicolson}
Using Eq. 4.13 of \cite{LeVeque2003}:
\begin{equation}\label{EqDiffCN_LV}
Q^{n+1}_i = Q^{n}_i + \frac{\Delta t}{2\Delta x^2} \left[ 
k_{I+1}\Delta Q^{n}_{I+1} - k_{I}\Delta Q^{n}_{I} +
k_{I+1}\Delta Q^{n+1}_{I+1} - k_{I}\Delta Q^{n+1}_{I} \right]
\end{equation}
or after modifying for curvilinear coordinates
\begin{equation}\label{EqDiffCN_LV}
Q^{n+1}_i = Q^{n}_i + \frac{\Delta t}{2\kappa_i^2} \left[ 
\left( k \sigma^2 \Delta Q^{n}  \right)_{I+1} - \left( k \sigma^2 \Delta Q^{n}  \right)_{I} +
\left( k \sigma^2 \Delta Q^{n+1}\right)_{I+1} - \left( k \sigma^2 \Delta Q^{n+1}\right)_{I} \right]
\end{equation}

\subsection{Numerical implementation}
