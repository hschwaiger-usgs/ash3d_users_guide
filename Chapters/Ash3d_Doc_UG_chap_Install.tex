%%%%%%%%%%%%%%%%%%%%%%%%%%%%%%%%%%%%%%%%%%%%%%%%%%%%%%%%%%%%%%%%%%%%%%%%%%%%%%%
\chapter{Installation}\label{ChapInstall}
Ash3d was developed to run on a linux system and has been successfully been installed
on Ubuntu and CentOS on x86\_64, Power8 and ARM systems (Pi3/4).
It has been successfully built on Mac OS 10 and Window 10, though implementing the
full functionality is on-going.

\paragraph{Library dependancies}
Ash3d requires three auxillary libraries:
\begin{itemize}
\item libMetReader.a  (\url{https://github.com/usgs/volcano-ash3d-metreader}) \\
This library is an interface between a calling program and a wide variety of
numerical weather prediction (NWP) models as well as radiosonde data.

\item libprojection.a (\url{https://github.com/usgs/volcano-ash3d-projection}) \\
This library converts between longitude/latitude coordinates and several projections
commonly used in NWP files.

\item libHoursSince.a (\url{https://github.com/usgs/volcano-ash3d-hourssince}) \\
This library converts dates and time values to the number of hours since a
reference year.
\end{itemize}

Each of these libraries were developed as a part of Ash3d, but were split into
separate libraries so as to support other software projects requiring an
interface to NWP models.  Instructions for building and installing these
libraries are given in the respective repositories.  The default installation
location for these libraries is \texttt{/opt/USGS}.  This could be changed to suit
your system by editing the makefile (\texttt{INSTALLDIR=\~/USGS} for example),
but should be consistent since the Ash3d makefile expects
a single location.  A minimal installation of these libraries only requires
a fortran compiler, however only ASCII windfiles (radiosonde or other profiles)
would be available.  To read NWP files from NCEP, NASA, WRF or otherwise, 
MetReader must be compiled with either NetCDF (preferably v4) or a GRIB
library (eccodes or the deprecated grib-api).

This can be installed for a CentOS (RHEL, Fedora) system by
\begin{verbatim}
     sudo yum install netcdf netcdf-devel eccodes eccodes-devel
\end{verbatim}
or for Ubuntu (Mint, Rasperian)
\begin{verbatim}
     sudo apt-get install netcdf netcdf-devel eccodes eccodes-devel
\end{verbatim}

In the MetReader library, it is encouraged to also download and install 
netcdf-java which is used to convert GRIB files to NetCDF
\url{http://artifacts.unidata.ucar.edu/content/repositories/unidata-releases/edu/ucar/netcdfAll/5.2.0/netcdfAll-5.2.0.jar}
To use this utility, you will need a java installation:
\begin{verbatim}
sudo yum install java-openjdk
\end{verbatim}
\begin{verbatim}
sudo apt-get install default-jdk
\end{verbatim}

Although not needed for a minimal Ash3d installation, the lapack and blas
libraries are required if the Crank-Nicolson scheme is selected for calculating
diffusion.\\

So to build the needed libraries, run:
\begin{verbatim}
cd /to/build/directory/
git clone https://github.com/usgs/volcano-ash3d-hourssince
cd volcano-ash3d-hourssince
vi makefile (only if you want to change INSTALLDIR=/opt/USGS)
make all
[sudo] make install

cd /to/build/directory/
git clone https://github.com/usgs/volcano-ash3d-projection
cd volcano-ash3d-projection
vi makefile (only if you want to change INSTALLDIR=/opt/USGS)
make all
[sudo] make install

cd /to/build/directory/
git clone https://github.com/usgs/volcano-ash3d-metreader
cd volcano-ash3d-metreader
vi makefile (only if you changed INSTALLDIR for the other libraries or disabled GRIB/NETCDF)
make all
[sudo] make install
\end{verbatim}

\paragraph{Building Ash3d}
Once the dependent libraries have been installed, Ash3d can be downloaded from
\url{https://github.com/usgs/volcano-ash3d}.

An autoconf installation process is in development, but currently,
compilation is controled through a manual editing of 
\texttt{src/makefile}.

The top block of makefile contains all the variables that the user should
need to set:
\begin{itemize}
 \item \texttt{SYSTEM = gfortran}\\
If you want to add blocks for differenct compiler flags, you can control
the paths through this variable.  Tested compilers include gfortran, ifort,
and g95.
 \item \texttt{RUN = OPT}\\
This variable allows easy switching among different compiler flags for debugging (DEBUG),
profiling (PROF), and optimized (OPT).  Additionally, OPTOMP can be used to include
openMP directives.
 \item \texttt{USGSROOT=/opt/USGS}\\
This is the path to where libhoursince, libprojection, and libMetReader are installed.
 \item \texttt{ASH3dCCSRC=./}\\
This is the location of the \texttt{src} directory.  The build directory can be
elsewhere.
 \item \texttt{INSTALLDIR=/opt/USGS/Ash3d}\\
This is the install path.
 \item \texttt{USENETCDF = T}\\
This variable is used to toggle (\texttt{T} or \texttt{F}) inclusion of netCDF
functionality.  MetReader would also need to be compiled with a consistant flag.
 \item \texttt{USEGRIB   = T}\\
This variable is used to toggle (\texttt{T} or \texttt{F}) inclusion of GRIB
functionality.  MetReader would also need to be compiled with a consistant flag.
 \item \texttt{VERB = 1}\\
This variable sets the level of debug information that is written to stdout.  Higher
numbers include more verbose output.
 \item \texttt{USEPOINTERS = F}\\
This variable allows some variables to be declared as allocatable pointers instead
of allocatable arrays.  Declaring variables as allocatable pointers allows an easier
interface with C programs (such as forestclaw).  The option to keep variables as
allocatable arrays is primarily for compatibility with some older compilers.  The
current implementation of OpenMP seems to require that pointers not be used.
This variable
will likely be deprecated in favor of exclusive allocatable pointer variables.
 \item \texttt{USEEXTDATA = F}\\
This variable allow (if \texttt{T}) some external data files to be used at run-time.
The default behavior is that a global airport list and a global list of volcanoes with
default eruption source parameters is read as a data variable at compile-time
to minimize the external files the executable needs at run-time.  Compiling
some subroutines with several thousand lines in which these data are stored can add
significantly to the compilation time and memory requirements, sometimes exceeding
the availble resources (such as on the Raspberry Pi 3).  If this variable is set to
\texttt{T}, the data files in \texttt{Ash3d/share} will be copied to the installtion
directory.  This path can be supplanted with another location using the optional
environment variable \texttt{ASH3DHOME}.
 \item \texttt{LIMITER = LIM\_SUPERBEE}\\
This variable controls which limiter is used in the advection routines.  The
Superbee limiter performs very well and is the default, however other limiters
can be used if desired (e.g. if linearity is needed).  Available options are:
no limiter (\texttt{LIM\_NONE}), Lax-Wendrof (\texttt{LIM\_LAXWEN}),
Beam-Warming (\texttt{LIM\_BW}), Fromm (\texttt{LIN\_FROMM}), 
MinMod (\texttt{LIM\_MINMOD}), SuperBee (\texttt{LIM\_SUPERBEE}),
and MC (\texttt{LIM\_MC}).
 \item \texttt{DIFFMETH=CRANKNIC}\\
This variable controls the integration scheme used for diffusion.  The default
is \texttt{CRANKNIC} which invokes the Crank-Nicolson semi-implicit scheme, but
this requires the  system libraries libblas and liblapack to be installed.
Alternatively, if set to \texttt{EXPLDIFF}, the explicit scheme is used.
\end{itemize}

Once these variables are modified to your system and preferences, Ash3d can
be built by typing \texttt{make}.  To install, type \texttt{make install}
either as root or with a installation path for which you have write
priviledges.

To uninstall, type \texttt{make uninstall}.
%%%%%%%%%%%%%%%%%%%%%%%%%%%%%%%%%%%%%%%%%%%%%%%%%%%%%%%%%%%%%%%%%%%%%%%%%%%%%%%
