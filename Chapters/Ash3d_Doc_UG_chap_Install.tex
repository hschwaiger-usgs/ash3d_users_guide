%%%%%%%%%%%%%%%%%%%%%%%%%%%%%%%%%%%%%%%%%%%%%%%%%%%%%%%%%%%%%%%%%%%%%%%%%%%%%%%
\chapter{Installation}\label{ChapInstall}
Ash3d was developed to run on a Linux system and has been tested on several
popular Debian and Redhat-based distributions, running on hardware as light
as Intel Atom processors or 32-bit ARM systems (Raspberry Pi 3), common
personal computer hardware, as well as modern high-performance computing
environments.
It has been successfully built on other Unix varieties such as OpenBSD
and Darwin (MacOS).  Experimental limited installations on Window 10/11 have been
successful, though implementing the full functionality is on-going.

\paragraph{Library dependancies}
Ash3d requires three auxillary libraries:
\begin{itemize}
\item libprojection.a (\url{https://code.usgs.gov/vsc/ash3d/volcano-ash3d-projection}) \\
This library converts between longitude/latitude coordinates and several projections
commonly used in NWP files.
This repository is mirrored at \url{https://github.com/DOI-USGS/volcano-ash3d-projection}.

\item libHoursSince.a (\url{https://code.usgs.gov/vsc/ash3d/volcano-ash3d-hourssince}) \\
This library converts dates and time values to the number of hours since a
reference year.
This repository is mirrored at \url{https://github.com/DOI-USGS/volcano-ash3d-hourssince}.

\item libmetreader.a  (\url{https://code.usgs.gov/vsc/ash3d/volcano-ash3d-metreader}) \\
This library is an interface between a calling program and a wide variety of
numerical weather prediction (NWP) models as well as radiosonde data.
This repository is mirrored at \url{https://github.com/DOI-USGS/volcano-ash3d-metreader}.

\end{itemize}

Each of these libraries were developed as a part of Ash3d, but were split into
separate libraries so as to support other software projects requiring an
interface to NWP models.  Instructions for building and installing these
libraries are given in the respective repositories.  The default installation
location for these libraries is \texttt{/opt/USGS}.  This could be changed to suit
your system by editing the \texttt{makefile}
(\texttt{INSTALLDIR=}$\sim$\texttt{/USGS} for example),
but should be consistent since the Ash3d \texttt{makefile} expects
a consistent location.  A minimal installation of these libraries only requires
a Fortran compiler, however only ASCII windfiles (radiosonde or other profiles)
would be available.  To read NWP files from NCEP, NASA, ECMWF, WRF or otherwise, 
MetReader must be compiled with either NetCDF (preferably v4) or the GRIB
library ecCodes.

This can be installed for a RedHat-based system (RHEL, CentOS, Rocky, Fedora) by
\begin{verbatim}
     sudo yum install netcdf netcdf-devel eccodes eccodes-devel
\end{verbatim}
or for Ubuntu (Mint, Rasperian)
\begin{verbatim}
     sudo apt-get install netcdf netcdf-devel eccodes eccodes-devel
\end{verbatim}

In the MetReader library, it is encouraged to also download and install 
\texttt{netcdf-java} which is used to convert GRIB files to NetCDF
\url{https://www.unidata.ucar.edu/software/netcdf-java}
To use this utility, you will need a java installation:
\begin{verbatim}
sudo yum install java-openjdk
\end{verbatim}
or for Ubuntu (Mint, Rasperian)
\begin{verbatim}
sudo apt-get install default-jdk
\end{verbatim}

Although not needed for a minimal Ash3d installation, the \texttt{lapack} and \texttt{blas}
libraries are required if the Crank-Nicolson scheme is selected for calculating
diffusion.\\

So to build the needed libraries, run:
\begin{verbatim}
cd /to/build/directory/
git clone https://github.com/DOI-USGS/volcano-ash3d-hourssince
cd volcano-ash3d-hourssince
  For local changes; vi makefile
make all
make check
[sudo] make install

cd /to/build/directory/
git clone https://github.com/DOI-USGS/volcano-ash3d-projection
cd volcano-ash3d-projection
  For local changes; vi makefile
make all
make check
[sudo] make install

cd /to/build/directory/
git clone https://github.com/DOI-USGS/volcano-ash3d-metreader
cd volcano-ash3d-metreader
  For local changes; vi makefile
make all
make check
[sudo] make install
\end{verbatim}

\paragraph{Building Ash3d}
Once the dependent libraries have been installed, Ash3d can be downloaded from
\url{https://code.usgs.gov/vsc/ash3d/volcano-ash3d}.

An autoconf installation process is in development, but currently,
compilation is controled through a manual editing of 
\texttt{src/makefile}.

The top block of makefile contains all the variables that the user should
need to set:
\begin{itemize}
 \item \texttt{SYSTEM = gfortran}\\
If you want to add blocks for different compiler flags, you can control
the paths through this variable.  Tested compilers include \texttt{gfortran},
\texttt{ifort}, and \texttt{aocc}. Compiler paths, flags, options are all
specified in include files with names \texttt{make\_[SYSTEM].inc}
 \item \texttt{RUN = OPT}\\
This variable allows easy switching among different compiler flags for debugging
(\texttt{DEBUG}),
profiling (\texttt{PROF}), and optimized (\texttt{OPT}).  Additionally, \texttt{OPTOMP}
can be used to include openMP directives.
 \item \texttt{OS=LINUX}\\
This variable specifies the operating system class used.  Default is \texttt{LINUX},but
can also be \texttt{MACOS} or \texttt{WINDOWS}. This is mainly used to direct the
software on how to build local paths and execute commands.
 \item \texttt{USGSROOT=/opt/USGS}\\
This is the path to where libhoursince, libprojection, and libmetreader are installed.
 \item \texttt{ASH3dCCSRC=./}\\
This is the location of the \texttt{src} directory.  The build directory can be
elsewhere.
 \item \texttt{INSTALLDIR=/opt/USGS/Ash3d}\\
This is the install path.
 \item \texttt{USENETCDF = T}\\
This variable is used to toggle (\texttt{T} or \texttt{F}) inclusion of NetCDF
functionality.  Metreader would also need to be compiled with a consistent flag.
 \item \texttt{USEGRIB   = T}\\
This variable is used to toggle (\texttt{T} or \texttt{F}) inclusion of GRIB
functionality.  Metreader would also need to be compiled with a consistent flag.
 \item \texttt{USEPOINTERS = F}\\
This variable allows some variables to be declared as allocatable pointers instead
of allocatable arrays.  Declaring variables as allocatable pointers allows an easier
interface with C programs (such as ForestClaw).  The option to keep variables as
allocatable arrays is primarily for compatibility with some older compilers.  The
current implementation of OpenMP seems to require that pointers not be used.
This variable
will likely be deprecated in favor of exclusive allocatable pointer variables.
 \item \texttt{USEEXTDATA = F}\\
This variable allows (if \texttt{T}) some external data files to be read at run-time.
The default behavior is that a global airport list and a global list of volcanoes with
default eruption source parameters are read as data variables at compile-time
to minimize the external files the executable needs at run-time.  Compiling
some subroutines with several thousand lines in which these data are stored can add
significantly to the compilation time and memory requirements, sometimes exceeding
the availble resources (such as on the Raspberry Pi 3).  If this variable is set to
\texttt{T}, the data files in \texttt{Ash3d/share} will be copied to the installtion
directory.  This path can be supplanted with another location using the optional
environment variable \texttt{ASH3DHOME}.
 \item \texttt{FASTFPPFLAG = }\\
This variable allow activating some limits on Ash3d computations which can significantly
speed-up calculations.  If \texttt{-DFAST\_DT} is used, the determination of the
maximum allowed time-step is limited to the time-steps of the NWP files, which are
generally at 1,3 or 6 hours.  Without this flag set, the maximum \texttt{dt} is
calculated each time-step.  This might be necessary for some processes such as
the radial spreading of umbrella clouds.  A second flag that can be used is
\texttt{-DFAST\_SUBGRID} which limits the flux calculations to just the bounding
box of the current ariborne ash cloud.
 \item \texttt{USEPLPLOT = F}\\
 \texttt{USEDISLIN = F}\\
 \texttt{USEGMT = F}\\
 These allow the inclusion of various plotting libraries to be build into the Ash3d
post-processing tools, allowing the direct creation of plots from \texttt{Ash3d\_PostProc}
 \item \texttt{LIMITER = LIM\_SUPERBEE}\\
This variable controls which limiter is used in the advection routines.  The
Superbee limiter performs very well and is the default, however other limiters
can be used if desired (e.g. if linearity is needed).  Available options are:
no limiter (\texttt{LIM\_NONE}), Lax-Wendrof (\texttt{LIM\_LAXWEN}),
Beam-Warming (\texttt{LIM\_BW}), Fromm (\texttt{LIN\_FROMM}), 
MinMod (\texttt{LIM\_MINMOD}), SuperBee (\texttt{LIM\_SUPERBEE}),
and MC (\texttt{LIM\_MC}).
 \item \texttt{DIFFMETH=CRANKNIC}\\
This variable controls the integration scheme used for diffusion.  The default
is \texttt{CRANKNIC} which invokes the Crank-Nicolson semi-implicit scheme, but
this requires the  system libraries \texttt{libblas} and \texttt{liblapack} to be installed.
Alternatively, if set to \texttt{EXPLDIFF}, the explicit scheme is used.
 \item \texttt{PII=ON}\\
Ash3d does an initial system check to determine the current working directory, location
of any shared files, tests of compiler (and version) and general computing
environment, including the user name and host name. If this flag is set to \texttt{ON}, this
information is collected and included in the output file.  This information
can be useful as archival metadata on run output, but it may also be considered insecure
to have usernames, hostnames and complete system paths. Set to \texttt{OFF} to
disable this logging.
\end{itemize}

Once these variables are modified to your system and preferences, Ash3d can
be built by typing \texttt{make}.  To install, type \texttt{make install}
either as root or with a installation path for which you have write
priviledges.

To uninstall, type \texttt{make uninstall}.
%%%%%%%%%%%%%%%%%%%%%%%%%%%%%%%%%%%%%%%%%%%%%%%%%%%%%%%%%%%%%%%%%%%%%%%%%%%%%%%
