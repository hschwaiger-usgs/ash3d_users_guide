\chapter{Variable Diffusivity}\label{ChapAppendVarDiff}
The resolution of the weather forecast (or reanalysis) data will not capture sub-grid
processes do to turbulent mixing in the atmosphere. To approximate this mixing, Ash3d
uses a diffusion term in the transport equation. A homogeneous and isotropic diffusivity
can be specified in the Ash3d control file, however it can be more realistic to 
treat the vertical and horizontal diffusivities separetely and to have the values
depend on local atmospheric conditions.

The basic approach is to recognize that atmosphere is stratified and can be divided into
several layers within which the fundamental length-scale of turbulent mixing varies.
Broadly speaking, the atmosphere can be divided based on the temperature structure,
with the lowest level (the troposphere) consisting of a temperature profile that
decreases with altitude. This layer is characterized by strong vertical mixing and
contains the majority of the mass of the atmosphere and nearly all the atmospheric
water, essentially all the `weather'. This decrease in temperature with height is
interupted by the presence of ozone which reacts with solar radiation in a process that
releases heat, causing the temperature profile to increase with altitude. This layer
with the increase in temperature with height (stratosphere) is characterized by
minimal vertical mixing.
The change in lapse rate of the temperature defines the boundary between these layers
and is called the `tropopause'. Above the stratosphere, the temperature again decreases with
altitude in the mesosphere, then once again increases with height in the thermosphere
where incoming solar radiation once again plays a significant role.

For the purposes of volcanic ash transport modeling, we focus on the troposphere and
its sublayers. The primary layer of interest within the troposphere is the bottom layer
of the atmosphere where conditions are 
is effected by the surface. This layer is called the Atmospheric Boundary Layer or
Planetary Boundary Layer and is generally several hundred to a few thousand meters thick.
The thickness depends on several factors such as the latitude or the solar heating and
radiative cooling of the surface, as well as the temperature profile.
The bottom 10\% of this Atomospheric Boundary Layer is called the `surface layer' and 
is generally well mixed, characterized by homogeneous fluxes of heat and momentum.

The approach we use in assigning a local diffusivity is by first identifying the location
of all these layers, then using the suitable expressions for calculating the local
diffusivity.

The primary driver of vertical mixing in the troposphere is the buoyant instability of a parcel
of air with respect to vertical perturbations in position. This can be quantified
by the Richardson number, which is the ratio of the production of turbulent
kinetic energy via buoyancy to the production due to vertical shear stress.
It is given by
%% Richardson number (gradient) : Stull Eq. 5.6.2 or Monin and Yaglom Eq. 6.50 or Jacobson Eq 8.42
\begin{equation}
\mathrm{Ri}_g = \frac{\frac{g}{\theta_v}\frac{\partial \theta_v}{\partial z}}
{\left( \frac{\partial u}{\partial z}\right)^2 + \left( \frac{\partial v}{\partial z}\right)^2}
\end{equation}
where $\theta_v$ is the virtual potential temperature and $u$ and $v$ are the horizontal velocities.

The virtual potential temperature is given by:
\begin{equation}
\theta_v = \theta \left[ 1 + \left( \frac{\mathrm{R}_{v}}{\mathrm{R}_{d}}-1\right) \, w_v \right] = \theta \left( 1 + 0.608 \, w_v \right)
\end{equation}
where $w_v$ is the mixing ratio of water vapor
(ratio of mass of water vapor to mass of dry air) % Stull 1.5.1b or Jacobson Eq 2.96
and $\mathrm{R}_{d}$ and $\mathrm{R}_{v}$ are the univerasal gas constants for dry air and vapor respectively.
The potential temperature $\theta$ is given by:
\begin{equation}
\theta = T \left( \frac{p_0}{p}\right)^{\mathrm{R}_{d}/c_p} = T \left( \frac{p_0}{p}\right)^{0.286}
\end{equation}
where $p_0$ is the reference pressure (1000 $\mathrm{mb}$), and $c_p$ is the specific heat at constant pressure.
The vertical profile in potential temperature is useful in quickly determining if a region of the
atmosphere is stable or not depending on if the lapse rate is super- or sub-adiabatic.

The Richardson number also serves this purpose.
The denominator of $\mathrm{Ri}$ is strictly positive (turbulence by shear flow) whereas the numerator can be either
positive or negative depending the vertical gradient of $\theta_v$. For a dry, stable, vertically
stratified atmosphere, $\frac{\partial \theta}{\partial z}=0$.
If $\frac{\partial \theta_v}{\partial z}<0$ (i.e. $\mathrm{Ri}_g<0$), then there is a super-adiabatic lapse rate
and atmosphere is unstable with strong convection. In adiabatic conditions where
$\frac{\partial \theta_v}{\partial z} \rightarrow 0$ (and $\mathrm{Ri}_g \rightarrow 0$),
there is only mechanical turbulence. If $\frac{\partial \theta}{\partial z}>0$, then there
is a sub-adiabatic lapse rate and the mechanical production of turbulence is diminished by
the temperature stratification up until the Richardson number reaches a critical value,
$\mathrm{Ri} = \mathrm{Ri}_c$, above which all mechanical turbulence is suppressed.

Near the Earth's surface, the surface layer consists of the region in which turbulent fluxes,
both heat and stress, are roughly constant (vary by less than 10\%). Generally, with the constant
fluxes, variables in this surface layer can be characterized by a similarity solution as a function
of a few dimensionless groupings of variables. This Richardson number can be considered a dimensionless
length by defining:
\begin{equation}
\mathrm{Ri} = \frac{z}{L}
\end{equation}
where $L$ is the Monin-Obukhov length which corresponds to the height where the buoyant energy production
equals the shear production. Because $\mathrm{Ri}$ can be either positive (stable) or negative (unstable), or
even 0 in perfectly neutral conditions where there is no buoyant turbulent energy production, $L$ is likewise
also either positive or negative, tending to $\infty$ in neutral conditions. Continuing with the assumptions
of similarity within the surface layer, a logarithmic velocity profile can be derived where
\begin{equation}
u(z) = \frac{u_{*}}{\kappa} \mathrm{ln} \frac{z-d}{z_0}
\end{equation}
where $d$ is the displacement height and $z_0$ is the roughness length.
$u_{*}$ is the friction velocity, which is a measure of the shear stress of the wind on the surface:
\begin{equation}
u_{*} = \sqrt{\frac{\tau}{\rho}}
\end{equation}





The basic
structure of the block consists of three input lines:
\small
\begin{verbatim}
yes 1                       # use horizontal variable diffusivity; model ID
yes                         # use vertical variable diffusivity
0.9                         # KH_SmagC
0.4                         # vonKarman
30.0                        # LambdaC
0.25                        # RI_CRIT
\end{verbatim}
\normalsize

\section{Horizontal Diffusivity}\label{ChapAppendVarDiff_Sec_Kh}
Smagorinski model.

\begin{equation}
K_h = C \Delta x \Delta y \sqrt{\left[ \frac{\partial u}{\partial x} -\frac{\partial v}{\partial y} \right]^2
+ \left[ \frac{\partial v}{\partial x} +\frac{\partial u}{\partial y} \right]^2}
\end{equation}

The first term in the radical is the ``horizontal tension'' term and the second is
the ``horizontal shearing strain'' term %\cite{Griffies00}.
As a whole, the radical has units of $\mathrm{t}^{-1}$.
The length scale should be proportional to the maximum wavenumber represented by the grid.
Since this is determined by the grid of the NWP data, the $\Delta x$ and $\Delta y$ are the 
corresponding horizontal length scales of the NWP data. Ash3d used the area of the cells
for this product. $C$ is a dimensionless scaling parameter.

Pielke model.
\begin{equation}
K_h = C \Delta x \Delta y \sqrt{
\frac{1}{2}\left[ \left(\frac{\partial u}{\partial x}\right)^2 + \left(\frac{\partial v}{\partial y}\right)^2 \right]
+ \left[ \frac{\partial v}{\partial x} +\frac{\partial u}{\partial y} \right]^2}
\end{equation}


\section{Vertical Diffusivity}\label{ChapAppendVarDiff_Sec_Kv}
Vertical structure:

\subsection{surface layer}
Monin-Obukhov length
\begin{equation}
L = \frac{\theta_v u^2_{*}}{\kappa g \theta_v*}
\end{equation}
Negative $L$ corresponds to unstable conditions, positive $L$ indicates a stable atmosphere. Neutral
conditions corresponds to $L=\infty$


Atmospheric Boundary Layer (diurnal)

Free atmosphere


\subsection{Atmospheric Stability Metrics}
%For the purpose of characterizing vertical diffusion,the primary term used to describe the stability
%of the atmosphere is the gradient Richardson number, $\mathrm{Ri}_g$,
%which is a non-dimensional number representing the
%balance of bouyancy production of turbulent kinetic energy to the production of turbulence by shear flow.
%It is given by
%% Richardson number (gradient) : Stull Eq. 5.6.2 or Monin and Yaglom Eq. 6.50 or Jacobson Eq 8.42
%\begin{equation}
%\mathrm{Ri}_g = \frac{\frac{g}{\theta_v}\frac{\partial \theta_v}{\partial z}}
%{\left( \frac{\partial u}{\partial z}\right)^2 + \left( \frac{\partial v}{\partial z}\right)^2}
%\end{equation}
%where $\theta_v$ is the virtual potential temperature and $u$ and $v$ are the horizontal velocities.
%The denominator is strictly positive (turbulence by shear flow) whereas the numerator can be either
%positive or negative depending the vertical gradient of $\theta_v$. For a dry, stable, vertically
%stratified atmosphere, $\frac{\partial \theta}{\partial z}=0$.
%If $\frac{\partial \theta_v}{\partial z}<0$ (i.e. $\mathrm{Ri}_g<0$), then there is a super-adiabatic lapse rate
%and atmosphere is unstable with strong convection. In adiabatic conditions where
%$\frac{\partial \theta_v}{\partial z} \rightarrow 0$ (and $\mathrm{Ri}_g \rightarrow 0$),
%there is only mechanical turbulence. If $\frac{\partial \theta}{\partial z}>0$, then there
%is a sub-adiabatic lapse rate and the mechanical production of turbulence is diminished by
%the temperature stratification up until the Richardson number reaches a critical value,
%$\mathrm{Ri} = \mathrm{Ri}_c$, above which all mechanical turbulence is suppressed.
In Ash3d, we approximate $\mathrm{Ri}_g$ with the bulk Richardson number.
% Richardson number (bulk) : Stull Eq. 5.6.3 or Jacobson Eq 8.39
\begin{equation}\label{VarDiff_Eq_Rib}
\mathrm{Ri}_b = \frac{g \Delta \theta_v \Delta z}
{\theta_v \left[ \left( \Delta u \right)^2 + \left( \Delta v \right)^2 \right]}
\end{equation}
The virtual potential temperature is given by:
\begin{equation}
\theta_v = \theta \left[ 1 + \left( \frac{\mathrm{R}_{v}}{\mathrm{R}_{d}}-1\right) \, w_v \right] = \theta \left( 1 + 0.608 \, w_v \right)
\end{equation}
where $w_v$ is the mixing ratio of water vapor
(ratio of mass of water vapor to mass of dry air). % Stull 1.5.1b or Jacobson Eq 2.96
The potential temperature $\theta$ is given by:
\begin{equation}
\theta = T \left( \frac{p_0}{p}\right)^{\mathrm{R}_{d}/c_p} = T \left( \frac{p_0}{p}\right)^{0.286}
\end{equation}
where $p_0$ is the reference pressure (1000 $\mathrm{mb}$), $\mathrm{R}_{d}$ is the universal
gas constant for dry air, and $c_p$ is the specific heat at constant pressure.
If \texttt{useMoisture=.true.} in the Variable Diffusivity input block, then Ash3d will attempt to
read liquid water mixing ratios from the provided NWP file.
If $w_v$ is not available but specific humidity ($q$) is, $w_v$ will be calculated from $q$ (ratio of mass of
water vapor to total mass of air) via the relation $w_v=q/(1-q)$.
If this is also not available or if \texttt{useMoisture=.false.}, then
the virtual potential temperature will be replaced by the potential temperature in Eq. \ref{VarDiff_Eq_Rib}.
$q$ typically does not exceed $0.02 \, \mathrm{kg}/\mathrm{kg}$ so the difference between $\theta_v$
and $\theta$ is generally less than 1\%.


\subsection{Atmospheric Boundary Layer}
\begin{equation}
K_z = \frac{u_{*} \kappa z}{\phi} \left( 1-\frac{z}{h_{BL}} \right)
\end{equation}

General form for $\phi(\zeta)$ where $\zeta = z/L$
\begin{eqnarray}\label{EqPhiB}
\phi &=& \left\{ \begin{array} {l@{\quad \quad}l}
 \left[ 1+\gamma \zeta\right]^{\alpha}  &:  \mathrm{if}\,\,\, \zeta < 0 \,\,\, \mathrm{unstable} \\
1                                       &:  \mathrm{if}\,\,\, \zeta = 0 \,\,\, \mathrm{neutral} \\
1 + \beta \zeta                         &:  \mathrm{if}\,\,\, \zeta > 0 \,\,\, \mathrm{stable}
\end{array}
\right.
\end{eqnarray}

Panofsky reports that an exponent $\alpha=-1/3$ is theoretically preferable
since $\phi$ follows $(z/L)^{-1/3}$ for large $-z/L$ (free-convection condition) Pan. p134.
$\beta$ should be from 4.7 to 5.2 (Pan. p136.).
%The turbulent Prandtl number, which is the
%ratio of the eddy diffusion coefficient for momentum to that of energy
%\begin{equation}
%\mathrm{Pr_t} = \frac{K_{m,zx}}{K_{h,zz}}
%\end{equation}
%should be about 0.74 when $\kappa=0.35$ or around 0.95 if $\kappa=0.4$.

\small
\begin{table}[htbp]
\begin{center}
\begin{tabular}{| c | c | c | c |}
\hline
Author & $\alpha$ & $\beta$ & $\gamma$\\
\hline
Businger-Dyer (1971)     & -1/4 & 5.0 & -16.0 \\
Ulke (2000)                & -1/2 & 9.2 & -13.0 \\
Carl (1973)                & -1/3 & 5.0 & -15.0 \\
Troen-Mahrt, (1986)            & -1/3 & 5.0 & -7.0  \\
\hline
\end{tabular}
\caption{\label{tab:ProjOpt}Ash3d projection options}
\end{center}
\end{table}
\normalsize



\subsection{Free Atmosphere}

\begin{equation}
K_z = \ell_c^2 \left| \frac{\partial V}{\partial z} \right|  F_c(\mathrm{Ri})
\end{equation}

\begin{equation}
\ell_c = \frac{\kappa z}{1+ \frac{\kappa z}{\lambda_c}}
\end{equation}


\begin{eqnarray}\label{EqPhiB}
F_c(\mathrm{Ri}) &=& \left\{ \begin{array} {l@{\quad \quad}l}
 \left[ 1 -18 \, \mathrm{Ri} \right]^{1/2}                         &: \mathrm{if}\,\,\, \mathrm{Ri} < 0 \,\,\, \mathrm{unstable} \\
 \left[ 1+ 10 \, \mathrm{Ri} \, (1+8 \, \mathrm{Ri}) \right]^{-1}  &: \mathrm{if}\,\,\, \mathrm{Ri} > 0 \,\,\, \mathrm{unstable}
\end{array}
\right.
\end{eqnarray}













