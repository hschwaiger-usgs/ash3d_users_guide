%%%%%%%%%%%%%%%%%%%%%%%%%%%%%%%%%%%%%%%%%%%%%%%%%%%%%%%%%%%%%%%%%%%%%%%%%%%%%%%
\chapter{Introduction}\label{ChapIntro}
In 2012, we introduced Ash3d, a volcanic ash transport and dispersion model
used to forecast ash-cloud movement and ash deposition during eruptions \cite{Schwaiger12}.
This was followed in 2013 by the introduction of a web interface that allowed any user
with an account to simulate ash-cloud transport and deposition \cite{Mastin13b}.
Subsequently, we have used Ash3d to study ash transport during recent
eruptions \cite{Mastin13a}, to evaluate the sensitivity of model inputs to
outcomes \cite{Mastin16}, examine effects of umbrella-cloud growth on eruption
dynamics \cite{Mastin14,Mastin20a},  study whether and how ash from New Zealand
supereruptions could be deposited in Antarctica \cite{Dunbar17}, develop
probabilistic tephra-hazard analyses for eruptions from Mount St. Helens \cite{Mastin20b}
and Taupo, New Zealand \cite{Barker19}, and examine Bayesian ensemble techniques
to improve ash-cloud forecast accuracy \cite{Denlinger12}.

Development of a web interface allowed users all over the world to use the model
without downloading and compiling the software or downloading large amounts of
numerical weather prediction (NWP) model data used to run the model.  The web
interface however provides only a small subset of functions available in Ash3d.
Specifying a grain-size distribution, or a time series of erupted pulses and
plume heights and eruption rates, or allowing for the growth of umbrella clouds,
has not been available due to our desire to keep things simple.  The software has
also not been available for download, meaning that only model developers and their
close collaborators have had access the the full suite of features.

With this release of Ash3d source code, users can take advantage of Ash3d's full
functionality.  Users with particular skill or interest may also advance the
source code to consider physical processes or numerical advances not yet envisioned.
The complexity of this model requires that it be installed by users who are versed
in Linux and comfortable running models at the command line.  Users who wish to use
current or forecast meteorological data may also have to have a Linux system that
can run scheduled NWP data downloads (tens of Gb daily) and other updates.  If you
have these skills and resources, and a desire to simulate tephra transport in a 3D
wind field, this code may be for you. Users need have no special knowledge in
numerical methods or modeling.

We begin with a brief overview of Ash3d, describe simulations using an ASCII input
file, and provide some guidance on post-processing the results. Use of Ash3d through
the USGS web-based interface is briefly summarized, but is more formally presented
in a companion document \cite{Mastin13a}. Ash3d is still in development, hence its
capabilities and documentation are likely to be revised in coming years.

\section{Model Overview}\label{ChapIntroSecModelOverview}
Ash3d calculates the transport of volcanic ash by dividing the atmosphere into a
three-dimensional grid of cells (Figure \ref{FigAsh3dGrid}) and calculating the
flow of mass through cell walls. During the eruption, tephra is injected at a constant
rate into the column of cells above the volcano (Figure \ref{FigAsh3dGrid}).
Using a 3-D time-dependent wind field imported from a numerical weather
prediction model, downwind advection and diffusion of ash is numerically calculated with a
diffusion rate determined by a user-specified diffusivity. Individual ash
particles fall at a rate determined by their settling velocity in air, and
deposit when they reach the ground surface.
\begin{figure}[htbp]
%\includegraphics[angle=90,scale=0.9]{asharrivaltimesairports_p1.pdf}
\parbox{15cm}{\caption{\label{FigAsh3dGrid}
Example model grids}}
\end{figure}

\subsection{Initial conditions}\label{ChapIntroSecInitCond}
Ash3d does not calculate the dynamics of a rising plume. Instead, it injects
tephra into a column of nodes above the volcano (Figure \ref{FigAsh3dGrid}).
Users may specify
that the ash be concentrated in a single cell, distributed evenly throughout
the column, distributed along a user-specified vertical profile, or distributed
vertically following the Suzuki equation
(Figure \ref{FigAsh3dGrid}) \cite{Carey96,Suzuki83}.
\begin{equation}
 \frac{\mathrm{d}S}{\mathrm{d}z} = S\frac{k^2\left( 1-z/H \right)\exp\left[k\left(z/H-1 \right) \right]}
 {H \left[1-\left(1+k \right)\exp \left( -k\right) \right]} \label{EqSuz}
\end{equation}
where $S$ is the total mass of erupted material in a given time step at a
given particle size, $H$ is the total plume height, $z$ is a given elevation
in the plume, and the shape factor $k$ is an adjustable constant that controls
ash distribution with height. A low value of $k$ gives a roughly uniform
distribution of mass with elevation, while higher values concentrate mass
near the plume top (See Figure \ref{FigSourceExample}).

For eruptions with a larger mass-eruption rate, limiting these source terms to just
the column above the vent can be inadaquate and the radial spreading of an
umbrella cloud must be accounted for. The radius of the umbrella cloud at a
particular time is given by:
\begin{equation}
 R = \left( \frac{3 \lambda N Q}{2 \pi}\right)^{\frac{1}{3}} t^{\frac{2}{3}}  \label{EqUmbRad}
\end{equation}
where the volume influx rate, $Q$, is given by:
\begin{equation}
 Q = C \sqrt{k_e} \frac{\dot{M}^{\frac{3}{4}}}{N^{\frac{5}{4}}} \label{EqUmbQ}
\end{equation}
The radial spreading velocity within the umbrella cloud ($r<R$) can be expressed as
\begin{equation}
% Costa formulation
 u(r) = \dot{R} \left( \frac{3 R}{4 r} + \frac{r}{4R}\right) \label{EqUmbRadVelCost}
\end{equation}
%\begin{equation}
%% Webster formulation
% u(r) = \dot{R} \sqrt{\frac{R}{r}} \label{EqUmbRadVelWebster}
%\end{equation}




\begin{figure}[htbp]
%\includegraphics[angle=90,scale=0.9]{asharrivaltimesairports_p1.pdf}
\parbox{15cm}{\caption{\label{FigSourceExample}
Source examples}}
\end{figure}

During an eruption, Ash3d injects tephra into these nodes at a constant
rate. The grain-size distribution inserted into each node in the column
is the same at each elevation and for each eruptive pulse.
Default values are described in Chapter \ref{ChapWebInterface}.

\subsection{Transport}\label{ChapIntroSecTrans}
Ash3d solves for the conservation of mass in each cell by tracking the mass
concentration $q$ with time $t$,
\begin{equation}\label{EqGovEqVect}
 \frac{\partial q}{\partial t} +
   \nabla \cdot \left[ \left(\mathbf{u} + \mathbf{v_s} \right) q \right]
 - \nabla \cdot \left( \mathbf{K} \nabla q\right) = S
\end{equation}
where $\mathbf{u}$ is the 3-D wind vector,
$\mathbf{v_s}$ the settling velocity, $K$ is the
diffusivity, and $S$ is a source term.  For most source types, $S$
is non-zero only in the column of nodes above the volcano.  The
exception is the umbrella cloud source term which includes a
$3 \times 3$ group of cells centered over the vent.
The settling velocity is determinted via \cite[p.182]{Bird60}
\begin{equation}
v_s=\sqrt{\frac{4d\rho_p g}{3C_d\rho_a}}\label{EqFallVel}
\end{equation}
where $d$ is the particle diamter, $\rho_p$ is the particle density, $g$ is
the acceleration of gravity, $C_d$ is the drag coefficient, and $\rho_a$
is the density of air.
Ash3d can use different models of $C_d$ to account for non-spherical particles.
By default, Ash3d calculates the settling
velocity of each particle size using equations of Wilson and Huang \cite{Wilson79}
for non-spherical particles.
\begin{equation}
C_d = \frac{24}{\mathrm{Re}}F^{-0.828}+2 \sqrt{1.07-F}\label{EqDragWH}
\end{equation}
$\mathrm{Re}$ is the Reynold's number and $F$ is a shape factor (see
Appendix \ref{ChapAppendFallDynamics} for more details.).
Alternative models that can be used include an
optional Cunningham slip correction factor \cite[p.407]{Seinfeld06},
which can increase the fall rate for small ($<30 \,\mathrm{\mu m}$)
particles at high elevation;
Stokes flow; and modifications to Wilson and Huang by \cite{Ganser93} and
\cite{Pfeiffer05}.

The advection term of Eq. \ref{EqGovEqVect} (the second term on the left-hand side
(LHS)) is calculated explicitly using the Donor Cell Upwind method of
solution (see \cite{Schwaiger12} for details). The diffusion term (third
LHS term) is calculated implicitly at the end of each time step. The diffusivity,
$K$, can be spatially variable and can be a function of the local meteorological
conditions. We have found that modeled clouds match well with observed ones
when $K=0$, and have set it to this value for simulations using the web interface.

\subsection{Deposition}\label{ChapIntroSecDepo}
Ash3d tracks the mass flux of each grain size across cell boundaries and
accumulates a deposit once tephra falls through the cell boundary that
represents the ground surface at a particular location.

